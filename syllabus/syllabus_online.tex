\documentclass[11pt]{article}

%\usepackage[cm]{fullpage}
\usepackage[lmargin=0.75in,rmargin=0.75in,
            tmargin=0.5in,bmargin=0.75in]{geometry}

\usepackage[parfill]{parskip}

\usepackage{hyperref}
\usepackage{xcolor}

\usepackage{palatino}

% URLs (special font for monospace)
\usepackage{inconsolata}
\usepackage[T1]{fontenc}

\usepackage[defaultsans]{cantarell}
\usepackage[T1]{fontenc}

\usepackage[small,compact]{titlesec}
\titlespacing{\subsection}{0pt}{*1}{-0.5\parskip}
\titleformat*{\subsection}{\sffamily\bfseries}

\usepackage{fancyhdr}

\pagestyle{fancy}

\fancyhead{}
\fancyfoot[LO,LE]{\sffamily \footnotesize PHY 546/Spring 2020}
\fancyfoot[CO,CE]{\thepage}
\renewcommand{\headrulewidth}{0.0pt}
\renewcommand{\footrulewidth}{0.0pt}

\newenvironment{itemsquish}
  { \begin{itemize}
    % set spacing between items
    \addtolength{\itemsep}{-0.25\baselineskip}
    % set spacing between lines
    \addtolength{\baselineskip}{-0.25\baselineskip} }
  { \end{itemize} }



\begin{document}

\begin{center}
{\LARGE \sffamily \bfseries PHY 546: Python for Scientific Computing} \\[3mm]
{\em Instructor}\/: Marivi Fernandez-Serra, PHY 139, maria.fernandez-serra@stonybrook.edu \\
{\em Date/Location}\/: Mondays, 3:00--3:53~pm \textcolor{red}{Online using Zoom}  \\
{\em Web:}\/ \url{https://marivifs-teaching.github.io/python-class/}
\end{center}

\subsection*{Learning Goals:}

The learning goal of this course is to enhance scholarship in an area
of active research, by learning how to apply python to problems
in your field.


\subsection*{Format:}

This course will mimic the original course developed by Professor
Mike Zingale.
The first few lectures will be delivered in a traditional format: the
instructor demonstrating some concepts via slides / interactive
examples.  After that, we will operated in a ``flipped'' manner.
Students will be expected to read and work through some basic
notebooks on their own before class, and then, in class, we will work
on projects together, sharing what we learn.

This class is {\em heavily} dependent on out-of-class discussion,
managed via slack.  All students are expected to participate in this
online discussion.

\textcolor{red}{Covid changes: As the course was already highly online,
the transition to a full online version is straight forward.
We will continue using Slack for out of class communication.
Lectures will be delivered using Zoom, but i will also use slack
simultaneously, as we have done so far.
I will continue recording attendance and collecting exercises with Slack}

\subsection*{Credit:}

This is a 1-credit course.


\subsection*{Contacting the Instructor:}

{\em e-mail:} maria.fernandez-serrae@stonybrook.edu ({add ``PHY 546'' to the
 start of the subject line of any e-mail}).
 
 \textcolor{red}{In our transition to a full online course
 Office Hours are maintained, but will be help online.
 A Zoom meeting will be organized upon request.}


\subsection*{Texts:}

There are no required textbooks for this class.  Online information
will be linked to from the course webpage.

A nice text for you to read along with the class, if desired, is
{\em Effective Computation in Physics}\/ by Scopatz \& Huff.



\subsection*{Lecture Topics:}

We will (try to) discuss the following topics in the course:
%
\begin{itemsquish}
\item {\em Introduction to python} (4 lectures) \\ data structures and
  control statements, functions, classes, popular modules, Jupyter
  notebooks

\item {\em Software engineering practices} (1 lecture) \\
  including {\tt git} and github, and unit testing (with nose)

\item {\em Introduction to the NumPy array library} (1 lecture)

\item {\em Pandas and the data frame} (1 lecture)

\item {\em matplotlib / bokeh / plot.ly for visualization} (1 lecture)

\item {\em SciPy and numerical methods} (2 lectures)

\item {\em Introduction to SymPy} (1 lecture)

\item {\em GUIs} (1 lecture)

\item {\em f2py and C extensions} (1.5 lectures)

\item {\em Building applications / packaging} (1 lecture)
\end{itemsquish}

\noindent Time-permitting, we will also discuss:
\begin{itemsquish}
\item {\em MayaVi for 3-d visualization}

\item {\em NetworkX}

\item {\em Interacting with forms on a webpage}

\item {\em Julia}
\end{itemsquish}

\noindent The actual course topics and time spent on each topic will depend on the
interest and the participation level of the class.


\subsection*{Computers:}
%
As we all use different systems for our research, we are not meeting
in a computer lab.  Instead, you should bring your own laptop to
class.  Information on how to install python on Windows, Mac OSX, and
Linux will be posted on the class webpage.

\noindent
If you don't have access to a computer that can run python, see me,
and I can try to arrange something for the semester.


\subsection*{Slack:}
%
We will use the slack team communication tool for all discussion.  You
will be added to the slack team at the start of the semester and then
can join in on the conversation at:
\url{https://sbu-phy546.slack.com}

Slack is organized into different channels, one for each major topic in class.


\subsection*{Evaluation:}


Students are expected to attend the class and to contribute
to the slack discussions (by asking questions, proposing examples, or
providing demonstrations of their own).  As we meet only one hour per
week, students show plan on spending time outside of class reviewing
and practicing the material we discussed.

\noindent {\em \bfseries The primary place for participation is the slack team chat}.  This is the place to interact with
me and your classmates---ask anything, share examples, etc.

\noindent Letter grades will be based on the online participation.  A
rough guide is presented below:
\begin{itemize}
\item {\sf A\phantom{+}}: 10 (meaningful) postings to slack {\bf plus}
  a short code example to our class git repo showing how you can apply what
  we've discussed in class to your field.

\item {\sf A$-$}: 10 (meaningful) postings to slack

\item {\sf B$+$}: 5 postings to the slack

\item {\sf B\phantom{+}}:  3 postings to the slack
\end{itemize}
A post does not mean a ``me to''-type post, but something either
demonstrating a problem you don't understand (giving code), asking for
some detail from the lecture to be explained, sharing a neat trick you
found, answering a classmate's question, etc.  {\em \bfseries Note that asking
questions about the content counts just as much as providing
answers}---the idea is to have a discussion outside of class on the
material.

\textcolor{red}{Covid Changes: The only change is that attendance to
the lectures will occur via zoom. Everything else does not
change because evaluation was already adapted for an online course}

\subsection*{Americans with Disabilities Act: }

\noindent If you have a physical, psychological, medical or learning
disability that may impact your course work, please contact Disability
Support Services, ECC (Educational Communications Center) Building,
Room 128, (631) 632-6748. They will determine with you what
accommodations, if any, are necessary and appropriate. All information
and documentation is confidential.



\subsection*{Academic Integrity: }

\noindent Each student must pursue his or her academic goals honestly
and be personally accountable for all submitted work.  Representing
another person's work as your own is always wrong.  Faculty are
required to report any suspected instances of academic dishonesty to
the Academic Judiciary. Faculty in the Health Sciences Center (School
of Health Technology \& Management, Nursing, Social Welfare, Dental
Medicine) and School of Medicine are required to follow their
school-specific procedures. For more comprehensive information on
academic integrity, including categories of academic dishonesty,
please refer to the academic judiciary website at
\url{http://www.stonybrook.edu/commcms/academic_integrity/}



\subsection*{Critical Incident Management: }

\noindent Stony Brook University expects students to respect the
rights, privileges, and property of other people. Faculty are required
to report to the Office of Judicial Affairs any disruptive behavior
that interrupts their ability to teach, compromises the safety of the
learning environment, or inhibits students' ability to learn.  Faculty
in the HSC Schools and the School of Medicine are required to follow
their school-specific procedures.



\subsection*{Electronic Communication: }

\noindent Email to your University email account is an important way
of communicating with you for this course.  For most students the
email address is `{\tt firstname.lastname@stonybrook.edu}'.
%, and the account can be accessed here.
{\em It is your responsibility to read your email received at this
  account.}  For instructions about how to verify your University
email address see this: \\[0.25em]
\url{http://it.stonybrook.edu/help/kb/checking-or-changing-your-mail-forwarding-address-in-the-epo}
\\[0.25em]
%
You can set up email forwarding using instructions here: \\[0.25em]
\url{http://it.stonybrook.edu/help/kb/setting-up-mail-forwarding-in-google-mail}
\\[0.25em]
%
If you choose to forward your University email to another account, we
are not responsible for any undeliverable messages.

\subsection*{Religious Observances: }

\noindent See the policy statement regarding religious holidays at
\\[0.25em] {
  \url{http://www.stonybrook.edu/registrar/forms/RelHolPol\%20081612\%20cr.pdf}}
\\[0.25em]
%
Students are expected to notify the course professors by email of
their intention to take time out for religious observance.  This
should be done as soon as possible but definitely before the end of
the `add/drop' period.  At that time they can discuss with the
instructor(s) how they will be able to make up the work covered.


\end{document}
